\section{Discussion}

\begin{comment}
\subsection{Feedback from the Medical Practitioner}

Mus mauris vitae ultricies leo integer malesuada nunc. Eget nullam non nisi est sit amet. Tristique magna sit amet purus gravida quis. Interdum posuere lorem ipsum dolor sit amet. Vestibulum morbi blandit cursus risus at. Sapien pellentesque habitant morbi tristique senectus et. Etiam tempor orci eu lobortis. Orci sagittis eu volutpat odio facilisis mauris. Tempus egestas sed sed risus pretium quam. Mi sit amet mauris commodo. Nibh cras pulvinar mattis nunc sed. Nec dui nunc mattis enim ut tellus elementum sagittis vitae. Mi tempus imperdiet nulla malesuada pellentesque elit eget. Molestie at elementum eu facilisis sed odio. Ut aliquam purus sit amet luctus venenatis lectus magna fringilla. Id donec ultrices tincidunt arcu non sodales neque sodales.

\end{comment}

\subsection{Extensibility to Remote Diagnosis}

The system initially assumed physical settings, such as hospitals, with a patient and a medical practitioner co-located. Yet, since the system fully runs online and may make use of the internet network, I believe that the system is extensible to remote clinic situations where each stakeholder is connected to collaborate remotely.

For example, once the patient learns how to use the system and run a test, and later be accustomed to managing the progress alone, the patient may carry out a test at home and upload the imagery to a server. Then, the practitioners may make clinical decisions remotely.

Plus, in such a way, I expect that the process may also be automated. For example, once the data is accumulated enough to generate a model, the change in the region in Amsler grid imagery may be calculated with such pre-trained models. In such a case, progress may easily be tracked and reported in a more manageable way, thus requiring less burden.

\subsection{Limitation \& Future Work}

Still, my study has several limitations that are required to be addressed.

First of all, I made design implications and implemented designs from the literature. Although the methodology seems valid as it follows the previous literature, it is also important to understand what the patients truly require toward an interactive Amsler grid app. Thus, an additional user study with real-world users might be needed to gain better insights.

Second, this system is yet to be tested in various computing devices. In order to test the system, I deployed the system on iPad 11-inch, Galaxy Tab, and MacBook 13-inch environments, respectively. However, even though the system is a universal web app, it is important to understand if any constraint on device specification exists. Thus, this app should be tested in more environments before distribution to forestall possible errors and malfunction.

Finally, this study adopted a heuristic evaluation method without evaluating the system with real-world users, which may not fully reveal the usability issues of users. In order to fully understand how patients perceive the system and gain feedback from them, clinical testing and interview sessions would be required.