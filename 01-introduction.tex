\chapter{INTRODUCTION}

Age-related macular degeneration (AMD) is a progressive chronic disease that is led by damage in the macula~\cite{lim2012age}. According to Jeon et al., AMD is a highly prevalent disease in society, where 6.62\% of the South Korean population are suffering from the symptoms of AMD~\cite{park2014age}. It is widely known that the symptom often accompanies disastrous symptoms, such as blurred vision or vision loss~\cite{lim2012age}. In addition to its detrimental effects on the vision itself, older adults with AMD are also prone to be affected psychologically, such as being depressed from the isolation from the society led by the symptom~\cite{rovner2007preventing}.

To date, however, little or no effective treatment for treating AMD exists~\cite{wong2011prevention}. Although some promising approaches such as stem cell implant have been proposed~\cite{carr2013development}, these methods are still yet to be widely applied. Thus, considering that it is almost impossible to revert once the degeneration is exacerbated, the most feasible and cost-effective management of AMD is to prevent further development of symptoms~\cite{al2017recent}, implying the necessity of the early detection and precise diagnosis of AMD.

As such, the importance of precise diagnosis has been highly emphasized so far. Yet, most of the diagnoses of AMD are limited in precisely reporting the region of issue. Specifically, patients and practitioners who conduct Amsler grid testing, the most prevalent AMD testing since the 1940s, often suffer from communicative issues, thus limiting the validity of the test~\cite{schuchard1993validity}. Heavily relying on the indirect, verbal report of patients, the regions of an issue are often inexactly reported, making such symptoms difficult to be accurately managed.

To address such an issue, I propose a novel application of self-reporting symptoms of AMD with Amsler grid. Specifically, based on the literature review of AMD and the symptoms that are led by the disease, I developed AmslerTouch, a touch-based Amsler-testing web app that supports patients to self-report AMD symptoms. AmslerTouch supports users to precisely annotate regions of symptoms. As such, I aimed to facilitate the decision-making of practitioners with the quantitative report of AMD symptoms with a widely available tablet device.

To evaluate the usability of AmslerTouch, I ran a heuristic evaluation with the checklist for heuristics. Based on the result of the evaluation and reflection on AmslerTouch, I also discuss possible enhancements and future works.